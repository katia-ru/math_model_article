\documentclass[a4paper]{article}
\usepackage[russian]{babel}
\usepackage[utf8x]{inputenc}
\usepackage{graphicx}
\usepackage{indentfirst}
\usepackage{cite}

\setcounter{tocdepth}{0}

\begin{document}

%\title{тайтл}
%УДК 577.322
%\author{Первый Автор}
%\author{Второй Автор}
%\maketitle

\begin{flushleft}
УДК 577.322
\end{flushleft}

\begin{center}
Статистические характеристики гидратных оболочек белков. Компьютерное моделирование.
\newline
И.О.Ф.
\newline
\newline
Аннотация.
\end{center}

Молекулы воды и сетка водородных связей являются необходимым структурным компонентом во многих биологических макромолекулярных системах, например, таких как нуклеиновые кислоты, фибриллярные и глобулярные белки. Известно, что плотность воды в гидратных оболочках белка отличается от плотности объёмной воды \cite{Svergun} . В данной работе мы рассматривали плотность как пространственную функцию удаления от белка. В ходе исследования были найдены характеристические особенности полученной функции плотности, а также дано объяснение указанным особенностям.
  
Ключевые слова: гидратная оболочка белков, молекулярная динамика, плотность 

\textbf{1. Введение.} Характерики физических свойств поверхностей “белок - растворитель” являются существенными для понимания структуры и фолдинга белков. Биологические молекулы растворены в воде, и взаимодействие с белками, аминокислотами и мембранами формирует их структуру и свойства \cite{Kauzman}.

В отсутствие окружающего водного слоя белки практически неподвижны и ферментативная активность пренебрежимо мала. В работах по дегидратации белков показано, что белок должен быть окружён по крайней мере первой гидратной оболочкой для поддержания своей функциональности \cite{Mattos}.

Молекулы воды в растворе белков можно разделить на 3 группы: 
\begin{itemize}
\item cильно и специфически связанная молекулы воды внутри белка (молекулы воды составляют элемент структуры)
\item молекулы воды, взаимодействующие с поверхностью белка
\item объёмная вода
\end{itemize}

Сильно связанные молекулы воды занимают внутренние полости белка и могут быть определены методами кристаллографии.  Поверхностные молекулы воды, образующие гидратную оболочку, подвижнее сильно связанных молекул, их водородные связи искажены, именно этот тип молекул исследуется в данной работе. 

Свойства слоя воды регулируются дипольным моментом, поляризацией и способностью каждой молекулы образовывать водородые связи. Водные молекулы взаимодействуют с макромалекулами на малых 
расстояниях ($\sim$ \AA). 

В экспериментальной работе \cite{Burling} с помощью рентгеновского и нейтронного рассеяния установлено, что на поверхности белка плотность воды значительно выше плотности объёмной воды. Также в ряде численных работ по моделированию установлено, что плотность слоя толщины N \AA на 15% превосходит плотность объёмной воды (N = 1 - 5) . Основная причина – это возмущение топологии сетки водородных связей воды \cite{Lee}. Это возмущение состоит в изменении внутренних параметров сетки: уменьшении расстояния О..О и увеличении координационнго числа молекулы воды.

Показано, что различия в плотности первой гидратной оболочки по сравнению с объёмной водой определяются топологическими и электростатическими свойствами поверхностей белков \cite{Mattos}. В среднем, более плотная вода находится в сжатом состоянии, при котором водные диполи стремятся выровняться параллельно друг другу электростатическим полем, генерируемым атомами белка.

Характеристики гидратации белков существенны для понимания их структуры и функций. Эти характеристики требуют выяснения взаимного влияния атомов белка и молекул воды. Для термодинамического описания гидратации белка – необходимо получить более глобальную картину, в которой растворитель описан в терминах вероятностных распределений. Известный методы для получения радиальных функций распределения в этом направлении – это молекулярное моделирование.

Внутренние параметры поверхностной воды значительно отличаются от объёмной. В работе \cite{Svergun} было с помощью комбинированного малоуглового рассеяния показано, что для различных белков средняя плотность первого гидратного слоя (0–3 Å от поверхности белка) выше плотности объёмной воды. Существуют работы, в которых данный факт подтверждается численными экспериментами (с помощью метода молекулярной динамики), а также методами кристаллографии.

В работе \cite{Levitt} объясняется физические причины этого эффекта с помощью численных экспериментов. Традиционно гидратация белка обсуждается в терминах  гидрофобности/гидрофильности поверхностных групп. Но в работe \cite{Cheng} указывается, что здесь имеет место изменения в поверхностной топологии. Изменение топологии сетки водородных связей, а так же электростатическое поле, генерируемое атомами белка, определяют повышенную плотность гидратной оболочки. Наиболее интересными для нас являются: функция распределения плотности и плотности электрического заряда в гидратных оболочках различных белков, а также сравнение полученных функций с аналогичными для объёмной воды.

\textbf{2. Проведение компьютерных экспериментов.} В данной работе проводилось изучение структур:
\begin{itemize}
\item 1CAG.pdb – тройная спираль коллагена, определённая с разрешением 1.9 \AA 
\item 1UBQ.pdb – белок убиквитин, определённый с разрешением 1.8 \AA
\item water ball -  шарообразное скопление молекул воды (3186 молекул), данная структура является аналогом объёмной воды
\end{itemize}

Для создания структуры water ball и гидратных оболочек белков была
использована программа SOLVATE \cite{solvate}.

После растворения изучаемых белков в воде необходимо привести систему в ненапряжённое состояние, сократить возможные влияния энергетически неоптимального расположения структуры “белок-растворитель”.  Для наших целей подходит такая последовательность процессов: 
\begin{enumerate}
\item минимизация потенциальной энергии структуры
\item дальнейшая молекулярная динамика при невысоких (приближенных к нормальным условиям, но с сохранением целостности структуры) температурах
\item для наших целей необходимо было взять несколько структур и сравнить их характеристики. Это необходимо для понимания: какие из особенностей статистических характеристик являются значимыми и воспроизводимыми, и какие – артефактами
\end{enumerate}

Одной из основных причин появления артефактов является достижение в процессе
минимизации методом градиентного спуска (применяемого в нашей работе) точки локального минимума $MIN_1$ (рис. *, точка А), но не точки глобального минимума $MIN_{ABS}$. Таким образом, в случае $| MIN_1 –- MIN_{ABS} | >> \epsilon$ (существенного различия локального и глобального минимумов) после минимизации система будет напряжена (обладать существенным запасом потенциальной энегрии), и последующая молекулярная динамика при различных начальных условиях будет приводить к различным статистическим характеристикам системы. Для уменьшения влияния этого факта необходимо чередовать минимизацию для преодоления барьера AB (рис. *).          


\addcontentsline{toc}{chapter}{Литература}
\begin{thebibliography}{99}
\bibitem{Svergun} D. I. Svergun, S. Richard, M. H. J. Koch, Z. Sayers, S. Kuprin, and G. Zaccai. Protein hydration in solution: Experimental observation by x-ray and neutron scattering // Proc. Natl. Acad. Sci. 1998. Т. 95. № 5. С. 2267–2272.
\bibitem{Kauzman} Kauzmann W. Some factors in the interpretation of protein denaturation. // Adv. Protein Chem. 1959. Т. 14. С. 1–63.
\bibitem{Mattos} Mattos C. Proteins in organic solvents // Curr. Opin. Struct. Biol. 2001. Т. 11. № 6. С. 761–764.


\bibitem{Burling} Burling F.T. и др. Direct Observation of Protein Solvation and Discrete Disorder with Experimental Crystallographic Phases // Science (80-. ). 1996. Т. 271. № 5245. С. 72–77.

\bibitem{Lee}  Lee S.H., Rossky P.J. A comparison of the structure and dynamics of liquid water at hydrophobic and hydrophilic surfaces—a molecular dynamics simulation study // J. Chem. Phys. 1994. Т. 100. № 4. С. 3334.

\bibitem{Levitt}  Levitt M., Sharon R. Accurate simulation of protein dynamics in solution. // Proc. Natl. Acad. Sci. U. S. A. 1988. Т. 85. № 20. С. 7557–61.

\bibitem{Cheng} Cheng Y.K., Sheu W.S., Rossky P.J. Hydrophobic hydration of amphipathic peptides. // Biophys. J. 1999. Т. 76. № 4. С. 1734–43.

\bibitem{cag} RCSB Protein Data Bank - RCSB PDB - 1CAG Structure Summary [Электронный ресурс]. URL: http://www.rcsb.org/pdb/explore.do?structureId=1CAG

\bibitem{ubq} RCSB Protein Data Bank - RCSB PDB - 1UBQ Structure Summary [Электронный ресурс]. URL: http://www.rcsb.org/pdb/explore/explore.do?structureId=1UBQ

\bibitem{solvate} Max Planck Institute for Biophysical Chemistry | Research | Research Groups | Theoretical and Computational Biophysics | Research | Methods | Solvate [Электронный ресурс]. URL: http://www.mpibpc.mpg.de/grubmueller/solvate

\end{thebibliography}

\end{document}