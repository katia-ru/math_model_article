\documentclass[a4paper]{article}
\usepackage[russian]{babel}
\usepackage[utf8x]{inputenc}
\usepackage{graphicx}
\usepackage{indentfirst}
\usepackage{cite}
\usepackage{enumerate}

\setcounter{tocdepth}{0}

\begin{document}

%\title{тайтл}

%УДК 577.3
%\author{Первый Автор}
%\author{Второй Автор}
%\maketitle

\begin{flushleft}
УДК 577.322
\end{flushleft}

\begin{center}
Статистические характеристики гидратных оболочек белков. Компьютерное моделирование.
\newline
И.О.Ф.
\newline
\newline
Аннотация.
\end{center}

Молекулы воды и сетка водородных связей являются необходимым структурным компонентом во многих биологических макромолекулярных системах, например, таких как нуклеиновые кислоты, фибриллярные и глобулярные белки. Известно, что плотность воды в гидратных оболочках белка отличается от плотности объёмной воды \cite{Svergun} . В данной работе мы рассматривали плотность как пространственную функцию удаления от белка. В ходе исследования были найдены характеристические особенности полученной функции плотности, а также дано объяснение указанным особенностям.
  
Ключевые слова: гидратная оболочка белков, молекулярная динамика, плотность 

\textbf{1. Введение.} Характерики физических свойств поверхностей “белок - растворитель” являются существенными для понимания структуры и фолдинга белков. Биологические молекулы растворены в воде, и взаимодействие с белками, аминокислотами и мембранами формирует их структуру и свойства \cite{Kauzman}.

В отсутствие окружающего водного слоя белки практически неподвижны и ферментативная активность пренебрежимо мала. В работах по дегидратации белков показано, что белок должен быть окружён по крайней мере первой гидратной оболочкой для поддержания своей функциональности \cite{Mattos}.

Молекулы воды в растворе белков можно разделить на 3 группы: 
\begin{itemize}
\item cильно и специфически связанная молекулы воды внутри белка (молекулы воды составляют элемент структуры)
\item молекулы воды, взаимодействующие с поверхностью белка
\item объёмная вода
\end{itemize}

Сильно связанные молекулы воды занимают внутренние полости белка и могут быть определены методами кристаллографии.  Поверхностные молекулы воды, образующие гидратную оболочку, подвижнее сильно связанных молекул, их водородные связи искажены, именно этот тип молекул исследуется в данной работе. 

Свойства слоя воды регулируются дипольным моментом, поляризацией и способностью каждой молекулы образовывать водородые связи. Водные молекулы взаимодействуют с макромалекулами на малых 
расстояниях ($\sim$ \AA). 

В экспериментальной работе \cite{Burling} с помощью рентгеновского и нейтронного рассеяния установлено, что на поверхности белка плотность воды значительно выше плотности объёмной воды. Также в ряде численных работ по моделированию установлено, что плотность слоя толщины N \AA на 15% превосходит плотность объёмной воды (N = 1 - 5) . Основная причина – это возмущение топологии сетки водородных связей воды \cite{Lee}. Это возмущение состоит в изменении внутренних параметров сетки: уменьшении расстояния О..О и увеличении координационнго числа молекулы воды.

Показано, что различия в плотности первой гидратной оболочки по сравнению с объёмной водой определяются топологическими и электростатическими свойствами поверхностей белков \cite{Mattos}. В среднем, более плотная вода находится в сжатом состоянии, при котором водные диполи стремятся выровняться параллельно друг другу электростатическим полем, генерируемым атомами белка.

Характеристики гидратации белков существенны для понимания их структуры и функций. Эти характеристики требуют выяснения взаимного влияния атомов белка и молекул воды. Для термодинамического описания гидратации белка – необходимо получить более глобальную картину, в которой растворитель описан в терминах вероятностных распределений. Известный методы для получения радиальных функций распределения в этом направлении – это молекулярное моделирование.

Внутренние параметры поверхностной воды значительно отличаются от объёмной. В работе \cite{Svergun} было с помощью комбинированного малоуглового рассеяния показано, что для различных белков средняя плотность первого гидратного слоя (0–3 Å от поверхности белка) выше плотности объёмной воды. Существуют работы, в которых данный факт подтверждается численными экспериментами (с помощью метода молекулярной динамики), а также методами кристаллографии.

В работе \cite{Levitt} объясняется физические причины этого эффекта с помощью численных экспериментов. Традиционно гидратация белка обсуждается в терминах  гидрофобности/гидрофильности поверхностных групп. Но в работe \cite{Cheng} указывается, что здесь имеет место изменения в поверхностной топологии. Изменение топологии сетки водородных связей, а так же электростатическое поле, генерируемое атомами белка, определяют повышенную плотность гидратной оболочки. Наиболее интересными для нас являются: функция распределения плотности и плотности электрического заряда в гидратных оболочках различных белков, а также сравнение полученных функций с аналогичными для объёмной воды.

\textbf{2. Проведение компьютерных экспериментов.} В данной работе проводилось изучение структур:
\begin{itemize}
\item 1CAG.pdb – тройная спираль коллагена, определённая с разрешением 1.9 \AA 
\item 1UBQ.pdb – белок убиквитин, определённый с разрешением 1.8 \AA
\item water ball -  шарообразное скопление молекул воды (3186 молекул), данная структура является аналогом объёмной воды
\end{itemize}

Для создания структуры water ball и гидратных оболочек белков была
использована программа SOLVATE \cite{solvate}.

После растворения изучаемых белков в воде необходимо привести систему в ненапряжённое состояние, сократить возможные влияния энергетически неоптимального расположения структуры “белок-растворитель”.  Для наших целей подходит такая последовательность процессов: 
\begin{enumerate}
\item минимизация потенциальной энергии структуры
\item дальнейшая молекулярная динамика при невысоких (приближенных к нормальным условиям, но с сохранением целостности структуры) температурах
\item для наших целей необходимо было взять несколько структур и сравнить их характеристики. Это необходимо для понимания: какие из особенностей статистических характеристик являются значимыми и воспроизводимыми, и какие – артефактами
\end{enumerate}

Одной из основных причин появления артефактов является достижение в процессе
минимизации методом градиентного спуска (применяемого в нашей работе) точки локального минимума $MIN_1$ (рис. \ref{ris:energy_barrier}, точка А), но не точки глобального минимума $MIN_{ABS}$. Таким образом, в случае $| MIN_1 –- MIN_{ABS} | >> \epsilon$ (существенного различия локального и глобального минимумов) после минимизации система будет напряжена (обладать существенным запасом потенциальной энегрии), и последующая молекулярная динамика при различных начальных условиях будет приводить к различным статистическим характеристикам системы. Для уменьшения влияния этого факта необходимо чередовать минимизацию для преодоления барьера AB.        

\begin{figure}[h]
\center{\includegraphics[width=0.5\linewidth]{energy_barrier.eps} }
\caption{ Зависимость потенциальной энергии системы, т. А - локальный минимум}
\label{ris:energy_barrier}
\end{figure}

Процессы минимизации и молекулярной динамики системы проводились с использованием программного пакет NAMD \cite{namd}

\textbf{2. 1 Проведение компьютерных экспериментов.} Была изучена плотность не только как усреднённая величина, относящаяся к системе множества молекул, но и как функция расстояния до поверхности белка.

Алгоритм нахождения плотности гидратной оболочки как функции минимального расстояния до молекулы белка:
\begin{enumerate}
\item С помощью пакета pyhull.py \cite{pyhull} приближаем оболочку водного слоя симплексами,
полученную оболочку обозначим как S
\item Находим $x_i^{min}$ и $x_i^{max}, i=1..3$ (т.е. находим внешних параллелепипед, внутри 
которого лежит структура «белок + вода»)
\item Определяем случайную точку $A_{RAND}$ в структуре (эта точка не обязательно имеет
координаты, совпадающие с каким-либо атомом кислорода).
\item Определяем, является ли точка внутренней, т.е. лежит ли она внутри оболочки S. Для этой
задачи используем алогоритм Möller–Trumbore \cite{moeller}.
\item Если точка внутренняя, то окружаем $A_{RAND}$  кубом $B_1..B_8$  с длиной ребра 1 таким, что длина ребра $B_1B_2$ = 1 \AA, $A_{RAND}$ - центр куба $B_1..B_8$. Предварительно проверяем, что данный куб
целиком лежит  в водном слое (является внутренней областью)
\item Плотность молекул воды в кубе: $\rho = N_{MOL}/V $, где $ N_{MOL} $ - количество атомов молекул
воды (считаем, что центр масс молекулы воды совпадает с координатами атома кислорода)
\item Под расстоянием по оси абсцисс будем понимать $ dist = min( r(A_{RAND}, C_{protein}), C_{protein} \in \Omega$, где $\Omega$ - множество точек белка. Таким образом получаем пары (x, y) = (dist, ro)
\item Повторяя итерации N раз (N >> 1), и усредняя значения ro для каждого dist – получаем
зависимость "средней плотности гидратной оболочки как функции.

\item Ошибку считаем следующим образом: пусть g(r) - функции распределения плотности. 
\begin{enumerate}[(a)]
\item Рассмотрим значения функции $ g_1 = g(r_n) $ и $ g_2 = g(r_{n+N}) $ – функции на
шаге n и (n+N) соответственно, где N – количество молекул воды в струкуре. N выбирается
равным количеству молекул воды, чтобы в среднем каждая молекула «оказала влияние» на
функцию $ g(r_{n+N}) $ 

\item Затем рассмотрим невязку по норме пространства $h_{[a, b]}: $
$\Delta g = 2/(\| g_1 \| + \| g_2 \|) * \sqrt {\int (g_1-g_2)^2 dr} $. Критерием прекращения итерационного процесса является достижение $| \Delta | < \epsilon$, где 
$\epsilon << 1$, в нашем случае $\epsilon = 0.001$. Физический смысл этого критерия: функция распределения плотности вышла на стабильный уровень, и с увеличением числа итераций – распределение значительно не меняется.


\end{enumerate}

\end{enumerate}

Алгоритм нахождения плотности объёмной воды как функции координаты r:
\begin{itemize}
\item данный алгоритм полностью повторяет описанный выше алгоритм для гидратной оболочки за исключением физического смысла оси абсцисс. 
\item пределим ось абсцисс: шар воды вписан в прямой параллелепипед. Найдём центр этого параллелепипеда $C_{CENTER}$ (точка пересечения диагоналей), под координатой $r$ будем понимать расстояние от $A_{RAND}$ до $C_{CENTER}$.
\end{itemize}

Функция распределения плотности для объёмной воды (рис. 1):

\begin{figure}[h]
\center{\includegraphics[width=0.5\linewidth]{DensityFunction_water_2.png} }
\caption{ Функция распределения плотности для объёмной воды}
\label{ris: water_density}
\end{figure}


\begin{figure}[H]
\begin{minipage}[h]{0.47\linewidth}
\center{\includegraphics[width=1\linewidth]{1UBQ_ver1.png}} a) \\
\end{minipage}
\hfill
\begin{minipage}[h]{0.47\linewidth}
\center{\includegraphics[width=1\linewidth]{third_1UBQ_8e+5_ver3.png}} \\b)
\end{minipage}
\vfill
\begin{minipage}[h]{0.47\linewidth}
\center{\includegraphics[width=1\linewidth]{1UBQ_1e+6_ver1.png}} c) \\
\end{minipage}
\hfill
\begin{minipage}[h]{0.47\linewidth}
\center{\includegraphics[width=1\linewidth]{1UBQ_ver3.png}} d) \\
\end{minipage}
\caption{ Функции распределения плотности в гидратных оболочках 1UBQ}
\label{ris: ubq_density}
\end{figure}

\begin{figure}[H]
\begin{minipage}[h]{0.47\linewidth}
\center{\includegraphics[width=1\linewidth]{1CAG_ver1.png}} a) \\
\end{minipage}
\hfill
\begin{minipage}[h]{0.47\linewidth}
\center{\includegraphics[width=1\linewidth]{1CAG_ver2.png}} \\b)
\end{minipage}
\vfill
\begin{minipage}[h]{0.47\linewidth}
\center{\includegraphics[width=1\linewidth]{1CAG_ver3.png}} c) \\
\end{minipage}
\hfill
\begin{minipage}[h]{0.47\linewidth}
\center{\includegraphics[width=1\linewidth]{third_1CAG_8e+3.png}} d) \\
\end{minipage}
\caption{ Функции распределения плотности в гидратных оболочках 1CAG}
\label{ris: ubq_density}
\end{figure}


\begin{figure}[H]
\begin{minipage}[h]{0.47\linewidth}
\center{\includegraphics[width=1\linewidth]{first_electr_1UBQ.png}} a) \\
\end{minipage}
\hfill
\begin{minipage}[h]{0.47\linewidth}
\center{\includegraphics[width=1\linewidth]{second_electr_1UBQ.png}} \\b)
\end{minipage}
\vfill
\begin{minipage}[h]{0.47\linewidth}
\center{\includegraphics[width=1\linewidth]{version_3_electr_1UBQ_third.png}} c) \\
\end{minipage}
\caption{ Функции распределения плотности отрицательных зарядов в гидратных оболочках 1UBQ}
\label{ris: ubq_density}
\end{figure}


\begin{figure}[H]
\begin{minipage}[h]{0.47\linewidth}
\center{\includegraphics[width=1\linewidth]{first_electr_1CAG.png}} a) \\
\end{minipage}
\hfill
\begin{minipage}[h]{0.47\linewidth}
\center{\includegraphics[width=1\linewidth]{second_electr_1CAG.png}} \\b)
\end{minipage}
\vfill
\begin{minipage}[h]{0.47\linewidth}
\center{\includegraphics[width=1\linewidth]{version_3_electr_1CAG_third.png}} c) \\
\end{minipage}

\caption{ Функции распределения плотности отрицательных зарядов в гидратных оболочках 1CAG}

\label{ris: ubq_density}
\end{figure}

Распределение отрицательного электрического заряда (плотность атомов кислорода) свидетельствует о неслучайном распределении молекул. Учитывая тот факт, что они находятся достаточно далеко от поверхности белка (10А – третья – четвертая(вторая-третья)? гидратная оболочка) можно говорить об электрическом и структурном упорядочивании...? Силовое поле гидратной оболочки определено структурой белка. Малые молекулы (для которых этот белок является ферментом), а также макромолекулярные структуры, присутствубщие в клетке в случае структурного белка реагируют именно на силовое поле. Возможно, заданное силовое поле гидратной оболочки является структурным паттерном как для структурных и регуляторных белков, так и для ферментов. Кулоновское дальнодействие гидратной оболочки, выстроенной благодаря водородным связям молекул воды и молекул белка, а также молекул воды между собой может обеспечивать ферментативное соответствие.


\addcontentsline{toc}{chapter}{Литература}
\begin{thebibliography}{99}
\bibitem{Svergun} D. I. Svergun, S. Richard, M. H. J. Koch, Z. Sayers, S. Kuprin, and G. Zaccai. Protein hydration in solution: Experimental observation by x-ray and neutron scattering // Proc. Natl. Acad. Sci. 1998. Т. 95. № 5. С. 2267–2272.
\bibitem{Kauzman} Kauzmann W. Some factors in the interpretation of protein denaturation. // Adv. Protein Chem. 1959. Т. 14. С. 1–63.
\bibitem{Mattos} Mattos C. Proteins in organic solvents // Curr. Opin. Struct. Biol. 2001. Т. 11. № 6. С. 761–764.


\bibitem{Burling} Burling F.T. и др. Direct Observation of Protein Solvation and Discrete Disorder with Experimental Crystallographic Phases // Science (80-. ). 1996. Т. 271. № 5245. С. 72–77.

\bibitem{Lee}  Lee S.H., Rossky P.J. A comparison of the structure and dynamics of liquid water at hydrophobic and hydrophilic surfaces—a molecular dynamics simulation study // J. Chem. Phys. 1994. Т. 100. № 4. С. 3334.

\bibitem{Levitt}  Levitt M., Sharon R. Accurate simulation of protein dynamics in solution. // Proc. Natl. Acad. Sci. U. S. A. 1988. Т. 85. № 20. С. 7557–61.

\bibitem{Cheng} Cheng Y.K., Sheu W.S., Rossky P.J. Hydrophobic hydration of amphipathic peptides. // Biophys. J. 1999. Т. 76. № 4. С. 1734–43.

\bibitem{cag} RCSB Protein Data Bank - RCSB PDB - 1CAG Structure Summary. URL: http://www.rcsb.org/pdb/explore.do?structureId=1CAG

\bibitem{ubq} RCSB Protein Data Bank - RCSB PDB - 1UBQ Structure Summary. URL: http://www.rcsb.org/pdb/explore/explore.do?structureId=1UBQ

\bibitem{solvate} Max Planck Institute for Biophysical Chemistry | Research | Research Groups | Theoretical and Computational Biophysics | Research | Methods | Solvate URL: http://www.mpibpc.mpg.de/grubmueller/solvate

\bibitem{namd} NAMD - Scalable Molecular Dynamics URL: http://www.ks.uiuc.edu/Research/namd/ 

\bibitem{pyhull} pyhull Package — pyhull 1.4.3 URL: https://pythonhosted.org/pyhull/pyhull.html

\bibitem{moeller} Moeller T., Trumbore B. Fast, Minimum Storage Ray-Triangle Intersection // J. Graph. Tools. 1997. Т. 2. № 1. С. 21–28.

\end{thebibliography}

\end{document}